%%%%%%%%%% Start TeXmacs macros
\newcommand{\tmtextit}[1]{{\itshape{#1}}}
%%%%%%%%%% End TeXmacs macros

\chapter*{Preface}

The book, of which I publish today the first part, is a summary of the
lectures I gave at the Sorbonne during the winters between 1882 and 1885. I
began my exploration of the \tmtextit{theory of surfaces} for the sole purpose
of finding new applications for the vast, yet little-known, theory of partial
differential equations. I had originally intended to spend barely one year on
this topic, but the interest of the subject, as well as the demands of my
students, led me far beyond the limits I had originally set.

This first volume is composed of three distinct parts. The first part deals
with the \tmtextit{geometric applications of relative motion}; I will return
to the propositions set out here in later sections when we study, in full
detail, the beautiful equations of M. Codazzi. The second part contains the
study of \tmtextit{various systems of curvilinear coordinates}. There I
examine, in turn, systems of conjugate lines (the study of which has been
greatly neglected), asymptotic lines, lines of curvature, and orthogonal and
isothermal systems.

The volume ends with the \tmtextit{theory of minimal surfaces}, wherein I make
use of the remarkable work published by the leading researchers of recent
years. It makes up roughly half of this present volume; with the exception of
the last three chapters, which were written shortly before publication, it was
taught on two different occasions, in 1882 and 1885. One or two important
issues have been omitted, and will be better placed in the next volume, where
I give the general propositions to which they relate.

As is his habit, M. Gauthier-Villars, after receiving this work has taken
every care with the printing, for which he receives my warmest thanks. I must
send them also to my students, who were eager to see these lectures published,
and particularly to one of our young researchers, M. G. Koenigs, lecturer at
the \tmtextit{\'Ecole Normale}, who kindly assisted me with proofreading.

\begin{flushright}
  {\small{June{\small{}} 14, 1887}}
\end{flushright}

\chapter*{Preface}
\addcontentsline{toc}{chapter}{Preface}

The book, of which this is the first part, is a summary of the lectures I gave at the Sorbonne
during the winters between 1882 and 1885. I began my exploration of the theory of surfaces for
the sole purpose of finding new applications for the vast, yet little-known, theory of partial
differential equations. I had originally intended to spend one year on this topic, but the
interest of the subject, as well as the demands of my students, led me far beyond the limits I
had originally set.

This first volume is composed of three distinct parts. The first part deals with the
\textit{applications of the geometry of relative motion}; I will return to the propositions set
out here in later sections when we study, in full detail, the beautiful equations of M. Codazzi.
The second part contains the study of \textit{various systems of curvilinear coordinates}. Here
I examine, in turn, systems of conjugate lines (whose study has been greatly neglected),
asymptotic lines, lines of curvature, and orthogonal and isothermic systems.

The volume ends with the \textit{theory of minimal surfaces}, wherein I make use of the
remarkable work published by the leading researchers of recent years. It makes up roughly half
of this present volume; except for the last three chapters, which were written shortly before
publication, it was taught on two different occasions, in 1882 and 1885. One or two important
issues have been omitted, and will be better placed in the next volume, where I give the general
propositions to which they relate.

After his usual habit, M. Gauthier-Villars has taken every care with the printing of this book,
for which he receives my warmest thanks. I send them also to my students, who were eager to see
these lectures published, and particularly to one of our young geometers, M. G. Koenigs, lecturer
at the \textit{Ecole Normale}, who kindly assisted me with proofreading.

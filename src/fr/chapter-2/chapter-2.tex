% !TEX root = ../master.tex
% !TEX encoding = UTF-8 Unicode

\Chapter{Sur L'Intégration Du Système Linéaire}{Qui Se Présente Dans La Théorie Précédente}
\label{chp2}

\textbf{13.} Il nous reste à étudier d'une manière détaillée l'intégration du système
\begin{empheq}[left=\empheqlbrace]{equation}
\begin{aligned}
\frac{\d{\alpha}}{\d{t}} &= \beta r - \gamma q, \\
\frac{\d{\beta}}{\d{t}} &= \gamma p - \alpha r \\
\frac{\d{\gamma}}{\d{t}} &= \alpha q - \beta p,
\end{aligned} \label{eqn-2.1}
\end{empheq}
auquel satisfont les trois groupes des cosinus. Nous avons déjà signalé une propriété fondamentale de ce système. Il 
admet l'intégrale du second degré

\begin{empheq}[left=\empheqlbrace]{equation}
\begin{aligned}
\alpha^2 + \beta^2 + \gamma^2 = \textrm{const.}
\end{aligned} \label{eqn-2.2}
\end{empheq}
et l'existence de cette intégrale entraîne, comme corollaires, une série de propositions qui facilitent, dans plusieurs 
cas, l'intégration du système.

Avant de commencer l'étude des équations (\ref{eqn-2.1}), je vais d'abord montrer que tout système linéaire de la forme
\begin{empheq}[left=\empheqlbrace]{equation}
\begin{aligned}
\frac{\d{\alpha}}{\d{t}} &= A\alpha + B\beta + C\gamma, \\
\frac{\d{\beta}}{\d{t}} &= A'\alpha + B'\beta + C'\gamma, \\
\frac{\d{\gamma}}{\d{t}} &= A''\alpha + B''\beta + C''\gamma,
\end{aligned} \label{eqn-2.3}
\end{empheq}
où $A$, $B$, $C$, \dots\, sont des fonctions de $t$, peut être ramené à la forme (\ref{eqn-2.1}) toutes les fois qu'il 
admet une intégrale du second degré
\begin{equation}
\varphi(\alpha, \beta, \gamma) = \textrm{const.},
\label{eqn-2.4}
\end{equation}
$\varphi$ désignant une fonction homogène du second degré, à coefficients constants ou variables.

En effet, par une substitution linéaire qui ne change évidement pas la forme des équations (\ref{eqn-2.3}), on peut 
ramener l'équation (\ref{eqn-2.4}) (sauf les cas exceptionnels, que l'on traitera facilement, où la fonction $\varphi$ 
serait un carré ou une somme de deux carrés) à la forme
\begin{equation}
\alpha^2 + \beta^2 + \gamma^2 = \textrm{const.}
\label{eqn-2.5}
\end{equation}
Si l'on exprime que le premier membre de cette équation est une intégrale du système (\ref{eqn-2.3}), on obtient les 
équations
\[
A = B' = C'' = B + A' = C + A'' = C' + B'' = 0,
\]
qui montrent bien que la système (\ref{eqn-2.3}) se ramène à la forme (\ref{eqn-2.1}).

Le système (\ref{eqn-2.1}) nous apparaît donc comme le type ou la \textit{forme réduite} d'une classe entière de 
systèmes présentant la propriété, que l'on rencontre fréquemment dans les applications, d'admettre une intégrale du 
second degré. Ce caractère particulier des équations que nous allons étudier méritait d'être signalé et suffirait à 
justifier l'étendue des développements qui vont suivre.

\textbf{14.} Je vais montrer d'abord que, toutes les fois que l'on connaîtra une solution particulière $(\alpha_0, 
\beta_0, \gamma_0)$ du système (\ref{eqn-2.1}), on pourra joindre l'intégrale du première degré
\[
\alpha\alpha_0 + \beta\beta_0 + \gamma\gamma_0 = \textrm{const.},
\]
à l'intégrale déjà donnée du second degré.

En effet, si l'on a une solution quelconque $(\alpha, \beta, \gamma)$ du système (\ref{eqn-2.1}), on en pourra déduire, 
d'après les propriétés de tout système linéaire, une solution plus générale
\[
\alpha + k\alpha_0, \quad \beta + k\beta_0, \quad \gamma + k\gamma_0,
\]
$k$ désignant une constante quelconque. On devra donc avoir, pour toutes les valeurs de $k$,
\[
(\alpha + k\alpha_0)^2 + (\beta + k\beta_0)^2 + (\gamma + k\gamma_0)^2 = \textrm{const.},
\]
ou, en développant,
\[
\alpha^2 + \beta^2 + \gamma^2 + 2k(\alpha\alpha_0 + \beta\beta_0 + \gamma\gamma_0) + k^2(\alpha_0^2 + \beta_0^2 + 
\gamma_0^2) = \textrm{const.}
\]

Le premier et le dernier terme du premier membre étant constants, il en sera de même de
\[
\alpha\alpha_0 + \beta\beta_0 + \gamma\gamma_0,
\]
comme il fallait de démontrer.

Il résulte évidemment de là que, si l'on connaissait seulement deux solutions particulières du système (\ref{eqn-2.1}), 
$(\alpha_0, \beta_0, \gamma_0)$, $(\alpha_1, \beta_1, \gamma_1)$, on pourrait immédiatement écrire la solution 
générale, qui serait définie par les équations
\begin{align*}
\alpha^2 + \beta^2 + \gamma^2 &= \textrm{const.}, \\
\alpha\alpha_0 + \beta\beta_0 + \gamma\gamma_0 &= \textrm{const.}, \\
\alpha\alpha_1 + \beta\beta_1 + \gamma\gamma_1 &= \textrm{const.};
\end{align*} 
ces équations peuvent être résolue et donnent pour $\alpha$, $\beta$, $\gamma$ les valeurs
\begin{empheq}[left=\empheqlbrace]{equation}
\begin{aligned}
\alpha &= c_0\alpha_0 + c_1\alpha_1 + c_2(\beta_0\gamma_1 - \beta_1\gamma_0), \\
\beta &= c_0\beta_0 + c_1\beta_1 + c_2(\gamma_0\alpha_1 - \alpha_0\gamma1), \\
\gamma &= c_0\gamma_0 + c_1\gamma_1 + c_2(\alpha_0\beta_1 - \alpha_1\beta_0),
\end{aligned} \label{eqn-2.6}
\end{empheq}
où $c_0$, $c_1$, $c_2$ désignent des constantes arbitraires. Mais on peut obtenir une proposition plus complète et 
montrer que, si l'on connaît une seule solution du système (\ref{eqn-2.1}), une seule quadrature suffira à nous donner 
son intégrale générale.

\textbf{15.} Pour établir ce résultat essentiel, remarquons que les valeurs les plus générales de $\alpha$, $\beta$, 
$\gamma$ doivent satisfaire à la relation
\[
\alpha^2 + \beta^2 + \gamma^2 = \textrm{const.}
\]

Commençons d'abord par écarter le cas où la constante serrait nulle; on pourra toujours, en divisant ces valeurs par 
une constant convenable, supposer que l'on a
\begin{equation}
\alpha^2 + \beta^2 + \gamma^2 = 1.
\label{eqn-2.7}
\end{equation}

Remarquons même que, dans le problème particulier que nous avons à traiter, $\alpha$, $\beta$, $\gamma$ étant trois 
cosinus directeurs, doivent nécessairement satisfaire à cette relation. Il est naturel d'exprimer 
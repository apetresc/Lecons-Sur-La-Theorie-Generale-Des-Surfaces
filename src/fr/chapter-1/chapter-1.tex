\Chapter{Du Déplacement A Un Paramètre}{Application A La Théorie Des Courbes Gauches}
\label{chp1}

\textbf{1.} Considérons un corps solide ou système invariable, mobile autour d'un point fixe. On sait qu'à un instant 
quelconque les vitesses des différents points du système sont les mêmes que s'il tournait autour d'une droite passant 
par le point fixe, droite qui a reçu le nom d'\textit{axe instantané de rotation}. On démontre en Mécanique que les 
rotations peuvent être représentées géométriquement par des droites, comme les forces, et composées ou décomposées 
suivant la même loi, c'est-a-dire que, si l'on compose ou si l'on décompose les rotations comme les forces, la vitesse 
imprimée par la rotation résultante à un point quelconque est la résultante des vitesses qui seraient communiquées au 
même point par chacune des rotations composantes, existant isolément. On sait aussi que, si l'on considère un point 
mobile par rapport au système invariable, la vitesse absolue de ce point est la résultante de sa vitesse relative et 
de sa vitesse d'\textit{entraînement}. On désigne sous ce nom la vitesse qu'aurait un point qui, à l'instant considéré, 
coïnciderait avec le point mobile, mais demeurerait invariablement lié au système solide.

Il résulte de ces propositions qu'on pourra construire, à un instant quelconque, les vitesses de tous les points du 
système invariable dès qu'on aura, en grandeur et en direction, la rotation à cet instant. Il semblerait naturel de 
déterminer à chaque instant cette rotation par ses composantes relatives à trois axes rectangulaires, fixes dans 
l'espace et ayant pour origine le point fixe du système solide. En réalité, les éléments les plus importants, les seuls 
qui permettent le plus souvent une étude approfondie due mouvement, ce sont les composantes de la rotation relativement 
à des axes mobiles, entraînés dans le mouvement du système invariable. Rappelons rapidement la méthode employée en 
Mécanique.

Soient OX, OY, OZ trois axes fixes passant par le point fixe O du système et O$x$, O$y$, O$z$ trois axes rectangulaires 
invariablement liés au système mobile. Nous supposerons que les deux systèmes d'axes aient la mêmes disposition, 
c'est-à-dire qu'ils puissent être amenés à coïncider. De plus, nous supposerons que les sens des axes aient été choisis 
de telle manière que la rotation autour de OZ, qui déplacerait OX du côté de OY, soit représentée par une droite 
dirigée suivant la partie positive de OZ. Nous déterminerons les axes mobiles par les cosinus des angles qu'ils forment 
avec les axes fixes. Pour cela nous écrirons le Tableau
$$
% TBD
Tableau
$$
qui fait connaître les cosinus des angles formés par chacun des axes fixes avec les axes mobiles.
 
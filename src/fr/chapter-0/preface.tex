%%%%%%%%%% Start TeXmacs macros
\newcommand{\tmtextit}[1]{{\itshape{#1}}}
%%%%%%%%%% End TeXmacs macros

\chapter*{Préface}

L'Ouvrage dont je publie aujourd'hi la première Partie est la résumé des Leçons que j'ai faites à la Sorbonne pendant 
les hivers de 1882 à 1885. J'avais commencé l'exposition de la \textit{Théorie des surfaces} dans le but unique d'y 
trouver des applications nouvelles de la théorie, si vaste et si peu connue, des équations aux dérivées partielles. Je 
comptais consacrer une année à peine à cet enseignement; mais l'intérêt du sujet, et aussi les demandes de mes 
auditeurs, m'on entrainé bien au delà des limites que j'avais primitivement fixées.

Ce premier Volume comprend trois parties distinctes. Le premier Livre traite des \textit{Applications à la Géométries 
de la théorie des mouvements relatifs}; j'aurai à revenir sur les propositions qui y sont exposées, dans la partie où 
seront étudiées plus tard, avec tous les détails nécessaires, les belles formules de M. Codazzi. Le second Livre 
contient l'étude des \textit{différents systèmes de coordonnées curvilignes}. J'y considère successivement les systèmes 
à lignes conjuguées, dont l'étude a été trop négligée, les lignes asymptotiques, les lignes de courbure, les systèmes 
orthogonaux et isothermes.

Le Volume se termine par le \textit{Théorie des surfaces minima}, où j'ai mis à profit les travaux si remarquable 
publiés par d'éminents géometres dans ces dernières années. Elle form à peu près la moitié de ce Volume; sauf les trois 
derniers Chapitres, qui ont été rédigés au moment de l'impression, elle a été enseignée à deux reprises diffèrentes, en 
1882 et 1885. Une ou deux questions importantes y ont été omises; elles seront mieux à leur place dans la suite, quand 
j'aurai donné les propositions générales auxquelles on peut les rattacher.

Suivant son habitude constante, M. Gauthier-Villars, après avoir accueilli cet Ouvrage, a apporté tous ses soins à 
l'impression; qu'il reçoive ici mes plus vifs remercîments; je dois aussi les adresser à mes auditeurs, qui ont désiré 
voir ces Leçons publiées, et, plus particulièrement, à un de nos jeunes géomètres, M. G. Koenigs, Maître de Conférences 
à l'École Normale, qui a bien voulu m'aider dans la revision des épreuves.

\begin{flushright}
  {\small{14 juin 1887.}}
\end{flushright}

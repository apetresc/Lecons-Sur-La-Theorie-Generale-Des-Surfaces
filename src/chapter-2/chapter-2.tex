\chapter{Solving Linear Systems of Differential Equations}
\label{chp2}

\textbf{13.} It remains to study in a detailed manner the solutions of systems of differential
equations of the form:
\begin{empheq}[left=\empheqlbrace]{align}
  \frac{\d{\alpha}}{\d{t}}  &= \beta  r - \gamma q, \nonumber \\
  \frac{\d{\beta}}{\d{t}}   &= \gamma p - \alpha r, \label{linear-ode} \\
  \frac{\d{\gamma}}{\d{t}}  &= \alpha q - \beta p   \nonumber
\end{empheq}


We have already encountered one of the fundamental properties of such a system: it admits a
quadratic solution of the form
\begin{align}
  \alpha^2 + \beta^2 + \gamma^2 = C,
\end{align}
for constant $C$, and the existance of this solution implies, as a corollary, a series of
propositions which enables, in many cases, the solution of the entire system.

Before beginning our study of \eqref{linear-ode}, we will first show that all systems of the form:
\begin{empheq}[left=\empheqlbrace]{align}
    \displaystyle \frac{\d{\alpha}}{\d{t}} &= A\alpha + B\beta + C\gamma, \nonumber \\
    \displaystyle \frac{\d{\beta}}{\d{t}}  &= A'\alpha + B'\beta + C'\gamma, \label{eqn-3}\\
    \displaystyle \frac{\d{\gamma}}{\d{t}} &= A''\alpha + B''\beta + C''\gamma \nonumber
\end{empheq}

where $A$, $B$, $C$ are functions of $t$, can be reduced to the form \eqref{linear-ode} whenever
they have a quadratic solution of the form:
\begin{align}
\varphi(\alpha, \beta, \gamma) = D, \label{eqn-4}
\end{align}
where $D$ is a constant and $\varphi$ is a homogenous quadratic function with arbitrary coefficients.

Indeed, since a linear homogenous substitution does not change the form of equations \eqref{eqn-3}, we
can use such substitutions to reduce \eqref{eqn-4} to the form
\begin{align}
\alpha^2 + \beta^2 + \gamma^2 = C, \label{eqn-5}
\end{align}
for constant $C$, leaving aside for the moment exceptional cases where $\varphi$ is a square
(or sum of squares), which we will deal with easily later on.

If we assume that the first term of \eqref{eqn-5} is a solution of \eqref{eqn-3}, we obtain the initial
conditions
\[
A = B' = C'' = B + A' = C + A'' = C' + B'' = 0,
\]
which allows us to reduce system \eqref{eqn-3} to the form \eqref{linear-ode}.

Thus, the system is revealed as a reduced form of an entire class of systems having the propery,
frequently encountered in practice, of admitting known solutions of the second degree. This unique
characteristic of the equations we will be studying is worth pursuing, and motivates the
developments which will follow.\footnote{We can also characterize such systems as 
\textit{self-adjoint}.}
